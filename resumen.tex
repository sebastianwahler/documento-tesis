% resumen en castellano
\thispagestyle{empty}
\chapter*{Resumen}

La presente tesis tiene como objetivo general estudiar las t�cnicas y metodolog�as actuales de an�lisis inteligente de datos y visualizaci�n para la asistencia en la investigaci�n criminal, a partir de los registros de actividades delictivas, sus autores y las relaciones de datos que puedan derivarse a partir de ellas. Es de especial inter�s la identificaci�n de redes ilegales, tales como bandas delictivas o criminales para propender a una persecuci�n penal inteligente.
Las actividades criminales habituales en una ciudad o regi�n van desde hurtos y robos de poca importancia, hasta otros de mayor gravedad como amenazas, cibercrimen, abusos sexuales y  homicidios. Todos ellos son registrados de diferentes formas por las fuerzas de la ley, con datos de variada precisi�n que incluyen usualmente la tipificaci�n del delito, los datos en tiempo y espacio, y en muchas ocasiones los autores correspondientes. Toda esta informaci�n respalda los procesos de investigaci�n judicial de cada caso, pero con el transcurso del tiempo constituyen una extensa base de conocimiento sobre la cual es posible extraer valiosa informaci�n para la prevenci�n del delito y la b�squeda de la justicia.  Por ejemplo, es posible identificar relaciones entre personas de acuerdo a un an�lisis transitivo de eventos criminales en tiempo y espacio que sugieren la conformaci�n de bandas delictivas. Las relaciones de amistad o conveniencia entre diversos autores de actividades criminales tambi�n puede inferirse de los registros delictivos y es de extrema relevancia para la prevenci�n del delito y la resoluci�n de casos inconclusos. Las Ciencias de la Computaci�n contribuyen aqu� en el an�lisis de datos y la inferencia automatizada de nueva informaci�n, as� tambi�n como en la representaci�n visual de esta informaci�n para facilitar su interpretaci�n y exploraci�n.
En este trabajo es de especial inter�s la aplicaci�n de estas t�cnicas y tecnolog�as utilizando los registros de actividades criminales de la Provincia de Chubut a trav�s de la colaboraci�n del Ministerio P�blico Fiscal de la provincia (parte del Poder Judicial con autonom�a funcional para la investigaci�n y persecuci�n de conductas delictivas) y las instituciones que lo asisten.

\bigskip

\noindent \textsc{Palabras Clave:} \textsf{}

\textit{Investigaci�n criminal}

\textit{An�lisis inteligente de datos}

\textit{Redes sociales}

\textit{Visualizaci�n}

\textit{Grafos}

\textit{Centralidad}

\textit{Comunidades}

\textit{PageRank}

%\textsf{fjisd}

